\documentclass{report}
\usepackage[T1]{fontenc}
\usepackage[utf8]{inputenc}
\usepackage[backend=biber, style=ieee]{biblatex}
\usepackage{csquotes}
\usepackage[portuguese]{babel}
\usepackage{blindtext}
\usepackage[printonlyused]{acronym}
\usepackage{hyperref}
\usepackage{graphicx}
\usepackage{indentfirst}
\usepackage{tabularx}
\usepackage{float}
\usepackage{subcaption}

%%\bibliography{bibliografia}
%%\setlength{\parskip}{0.39em}
%%\newcommand{\centered}[1]{\begin{tabular}{l} #1 \end{tabular}}


%For the header and footer
\usepackage{fancyhdr}
\fancypagestyle{plain}{%
\fancyfoot[L]{\emph{Departamento de Eletrónica, Telecomunicações e Informática }} % except the center
\fancyfoot[R]{\thepage{}}
\renewcommand{\headrulewidth}{0.4pt}
\renewcommand{\footrulewidth}{0.4pt}
}
\pagestyle{fancy}
\rhead{\emph{Projeto Final LABI2022G5}}
\fancyfoot[LO,L]{\emph{Departamento de Eletrónica, Telecomunicações e Informática }}
\cfoot{}
\fancyfoot[RO, RE]{\thepage}
\renewcommand{\headrulewidth}{0.1pt}
\renewcommand{\footrulewidth}{0.1pt}
%For the header and footer Over

%Page Border
\usepackage{pgf}
\usepackage{pgfpages}

\pgfpagesdeclarelayout{boxed}
{
  \edef\pgfpageoptionborder{0pt}
}
{
  \pgfpagesphysicalpageoptions
  {%
    logical pages=1,%
  }
  \pgfpageslogicalpageoptions{1}
  {
    border code=\pgfsetlinewidth{2pt}\pgfstroke,%
    border shrink=\pgfpageoptionborder,%
    resized width=.95\pgfphysicalwidth,%
    resized height=.95\pgfphysicalheight,%
    center=\pgfpoint{.5\pgfphysicalwidth}{.5\pgfphysicalheight}%
  }%
}
\pgfpagesuselayout{boxed}
\setlength{\parindent}{1cm}




\begin{document}
%%% DEFINIÇÕES GLOBAIS %%%
\def\titulo{Projeto Final LABI2022}
\def\data{Aveiro, julho 2021}
\def\autores{Miguel Vila, Diogo Silva, Miguel Reis, Martim Carvalho}
\def\autorescontactos{(107276) \href{mailto:miguelovila@ua.pt}{miguelovila@ua.pt}, (108212) \href{mailto:dsgps@ua.pt}{dsgps@ua.pt}, (108545) \href{mailto:mreis@ua.pt}{mreis@ua.pt}, (108749) \href{mailto:martimhcarvalho@ua.pt}{martimhcarvalho@ua.pt}}
\def\projetocodeua{\url{http://code.ua.pt/projects/labi2022g5}}
\def\departamento{DETI}
\def\empresa{UNIVERSIDADE DE AVEIRO}
\def\logotipo{ua.pdf}

%%% ESTRUTURA CAPA %%%
\begin{titlepage}
\begin{center}
\vspace*{20mm}
{\Huge \titulo}\\ 
\vspace{10mm}
{\Large \empresa}\\
\vspace{10mm}
{\LARGE \autores}\\ 
\vspace{30mm}
\begin{figure}[h]
\center
\includegraphics{\logotipo}
\end{figure}
\vspace{20mm}
\end{center}
\begin{flushright}
\end{flushright}
\end{titlepage}

%%%  PÁGINA DE TÍTULO %%%
\title{%
{\Huge\textbf{\titulo}}\\
{\Large \departamento\\ \empresa}
}
\author{
    \autores\\\\\
    \autorescontactos\\\\\\\
    \projetocodeua\\
}
\date{\data}
\maketitle
\pagenumbering{roman}

%%% OBJETIVOS %%%
\renewcommand{\abstractname}{OBJETIVOS DO PROJETO}
\begin{abstract}

Neste trabalho foi nos pedido para elaborar um site online com o objetivo dos seus utilizadores poderem armazenar as suas imagens de forma rapida de intuitiva. 
Alem de poderem armazenarem as imagens, os utilizadores poderao tambem publicar no site as suas proprias criacoes, formando uma colecao visivel a qualquer utilizador, e disponivel para ser requisitada por qualquer um que esteja registado.
Todas as imagens colocadas no site serao sujeitas a uma marca de agua gerada pelo site, como o logo do mesmo com o objetivo de garantir a autenticidade das mesmas.
Tal como foi dito, cada imagem tem a sua marca de agua, onde estara o logo do site e o nome do usuario que a possui, para que nao surjam copias e garantindo o valor das mesmas. Para as imagens poderem ser trocadas pelos diversos utilizadores, foi tambem criado um sistema de envio das imagens, basta aceder a sua colecao e, ao selecionar a imagem devera introduzir o nome de utilizador a quem deseja enviar a imagem e, instantaneamente estara na colecao da outra pessoa.
\end{abstract}



\end{document}